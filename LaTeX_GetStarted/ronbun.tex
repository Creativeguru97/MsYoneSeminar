%
% このファイルはヒューマンインタフェース研究報告集用クラスファイル
% hisken.cls(ver.1.0) を利用した「原稿執筆の手引き」です.
% Revised in Jun. 7, 2007 by Yutaka Ishii
%

\documentclass[a4paper,dvipdfmx]{hisken}
%\usepackage[dviout]{graphicx}
%\usepackage[dvipdfm]{graphicx}
\usepackage{latexsym}
\usepackage{graphics}
\usepackage{epsfig} \let\epsfile=\epsfig
\usepackage{url}

%和文タイトル
\jtitle{環境音を利用した, 孤独感を軽減するエージェントの研究}

%著者日本名
\jauthor{
梅田 和希\thanks{関西大学 総合情報学部}   
}
%英文タイトル
\etitle{Research of virtual sound agent decrease feeling of loneliness}

%著者英文名
\eauthor{
Kazuki Umeda\thanks{Kansai University, Faculty of Informatics},
}
% 名前は,はじめの一文字だけ大文字にしてください.

\begin{document}

%maketitle は abstract と keyword の後に入れてください.

\begin{abstract}
本研究では,ARを用いたアニメーションを開発し,史実の流れや関係性を視覚的に提示して中等教育の歴史の理解を助ける学習支援を行う.本システムでは,歴史教育の中でも特に江戸時代前後に発生した戦いに関する史実に重点を置き,AR,3次元CGによる地図上の史実の視覚化と,スライドバーによる時系列ブラウジングシステムを提案する.これにより,あらゆる歴史教育用コンテンツをブラウジングできるように記憶定着することを狙う.CGは,各史実に関係性のある人物や建築物を示す際に用いる.また,スライドバーは,史実の年代ごとのアニメーションを時間ブラウジング操作できるように実装した.中等教育で学習する史実の知識を定着させる効果を予想し被験者実験を行ったところ,○○という結果が示された.
\end{abstract}

\begin{keyword}	
%%History, augmented reality, 3DCG, animation, learning support
歴史, AR, 3次元CG, アニメーション, 学習支援
\end{keyword}

\maketitle
	
\section{はじめに}
\subsection{ARの現状}

近年,拡張現実感augmented reality(以下:AR)がエンターテインメントや通信販売など,様々なサービスで用いられている\cite{manabe}.ARとは,ユーザが見ている現実の場面にコンピュータグラフィクス(CG)によって描かれた仮想物体を重畳表示することで, ユーザの場所に応じた情報を直感的に提示する技術のことである\cite{kanbara}.今やARは1960年代に使用されていたHMD(Head-Mounted Display)型の作業支援システム「Ultimate Display」\cite{display}から消費者向けのシステムへと,様々な場面で利用されるように発展した\cite{rekimoto}.モバイル環境に特化したものとして,エンターテインメント向けに任天堂株式会社\footnote{\url{http://www. nintendo. co. jp/}}が開発したスマートフォン用アプリケーション「Pok\'{e}mon GO」\cite{hoshi}や,通信販売向けにInter IKEA Systems B.V.\cite{fran} が開発したスマートフォン用アプリケーション「IKEA Place」\cite{ikea}などは一般ユーザを対象に提供されている.

ARの教育適用については,中等教育までの学習に電子端末などのディジタル技術は導入が進まず\cite{ifip}\cite{ogawa}\cite{akahori},ARや電子黒板などを教育現場で利用している事例はあまり多くない\cite{okumura}\cite{kokuban}.この原因としてディジタル技術を使用する教師への教育不足\cite{suzuki}やPC・ソフトの導入コスト\cite{2005},黒板との操作性やインターフェース面の違い\cite{environment},今までの教授法に満足した教員によるチャレンジ意欲の欠如\cite{koizumi}が考えられる.一方で,近年タブレットやスマートフォン端末を利用した学習・教育システムが導入されつつあり\cite{hasegawa},生徒たちの学力向上に効果が出ている\cite{ict}.特に,スマートフォン端末などの電子端末を用いた学習では,意欲・関心・態度という観点で最も効果が確認された.また,授業の導入段階による電子端末活用は,知識・理解の観点で効果的であると確認された.本研究では,電子端末を利用した学習支援システムとした.そして,特定のアプリケーションやソフトウェアを必要としないWebアプリケーションを用いて,汎用的に使用できる学習支援システムを提案する.特に中学校で実際に歴史の授業を受講したことがある学生を対象に,中等教育で学習する史実の知識を定着させることを目的とした,時系列ブラウジング形式の歴史学習支援システムを提案した.

\subsection{中等教育における歴史について}

中等教育の授業から「社会」の授業が,「地理」,「歴史」,「公民」と3つの科目に分かれるため\cite{takezawa},学生たちは必然的に勉強時間が増えてしまい,時間的余裕や学習意欲が低下してしまう\cite{okano}.本研究では,地理的知識と歴史には関係性があると考え\cite{toida}\cite{terao},それらを同時に学ぶためのスキームが必要だと考えた.中等教育における歴史の教育では,知識の記憶と定着に焦点を置いている.知識の記憶とは,史実の把握と意味の理解,さらには史実間の関係性や特徴または背景を把握することである\cite{okano}.実際に中等教育の歴史の授業で使用されている教科書には,索引欄に約1300語の用語が記載されている\cite{rekishi}ものもあり,史実を初めて学ぶ学習者が知識として定着させることは困難であると思われる.本研究は,提案システムによって知識の定着を簡易化する「歴史」の学習支援を図る.

本システムの具体的手法として,このシステムには3次元CGやARで表現したアニメーションを用いる.アニメーションを取り入れることで,史実の意味や特徴,さらには他の史実との関係性などを学生たちに分かりやすくかつ詳細に示せると考えた.アニメーションには日本地図や史実の中心となった人物の肖像画を取り入れ,物語の年代に合わせてアニメーション内の色が一部変化するように設計した.また,アニメーションの切り替えと進行にスライドバーを用いることで,直感的に操作できるように実装した.これにより,史実の意味だけでなくその時代における勢力図や時代の流れまでをより鮮明に把握できると期待する.また,3次元CGを取り入れることで戦国武将に立体的な形を与えることができるため,親近感や躍動感などの強い印象を与えることができると考えた.また,ARを取り入れることで,従来のテキスト学習にディジタル技術を付加した立体的な観点で学習でき,アニメーションの内容や3次元CGの位置関係などの視覚的な情報が増えることで人々の記憶により定着すると考えた\cite{hujimoto}.本稿では,これらの技術を用いたサービスを提案した.


\section{関連研究}
\subsection{記憶術}

藤本ら\cite{hujimoto}は,ARを用いて情報を提示した際に,位置によってユーザが記憶できる情報量に違いがあるかを検証した.その際,「場所法」\cite{brien}というあらかじめ設定した場所に暗記事項を順番に置いていくイメージをしながら記憶していく記憶術の有効性を検証するために,2次元の紙面に独立した無意味な記号を複数配置した状態で,表示位置を変化させて記憶に影響があるかを確認する実験を行った.その結果,対象の位置と関連付けて表示する際,特に位置が重要である場合において,ARを用いた注釈情報は記憶に有効であると示された.これにより,ARを用いて特定の位置に情報を提示することで,「場所法」が適用された時と同等の状態となり,史実の情報を記憶により強く定着させることができると考えられる.

Mark.Billinghurst\cite{magic}らは,Magic Bookという実在の本とバーチャル空間を使ったMixed Realityインターフェースを開発し,現実世界と仮想世界の間に存在する不連続性を取り除くことで,拡張現実や仮想現実などの技術の重要性を示した.通常の書籍を主要なインターフェースオブジェクトとして使用し,書籍のページをめくりながら,ARマーカを認識して表示する3Dモデルを変えることができる.さらに,HMDを用いたVRビューにより没入的表示を実現し,自由に場面を移動しながらストーリー内の仮想モデルと対話できるようになり,ユーザは実際にストーリーの一部となることができる.また前川ら\cite{maekawa}は平仮名1文字のARマーカを用いたストーリ作成型ARひらがな学習システムを提案した.この研究は,複数のARマーカを組み合わせて単語を作成することで,その単語が意味する3DCGがARマーカ上に映されるというシステムである.また,複数の単語を作成した際に,特定の組み合わせであれば,3DCGが新たに動き出す処理も加えている.この研究では,使いやすさ,学習効率,面白さという3つの項目を検証するためにARマーカを組み合わせて単語を作成するという実験を行った.結果として,3DCGを表示させることで興味を持つことや3DCG同士の関係性や設定環境によって動作が変化することが学習に有効的だったと示されている.そのため,物語を白紙の本の上に作成していくプロセスで登場人物である動物のひらがなを記憶することが期待される.
これらは,「ストーリーテリング法」によるという記憶術\cite{oaks}に基づくシステムの検討だと考えられる.

本稿では,物語性と進行度の操作に着目し,進行度を操作できるアニメーション形式の史実学習コンテンツとして,学習効果が期待できると考えられる.


\subsection{学習支援システム}

学習支援システムとは,LMS(Learning Management System)\cite{tsuji}やe-learning\cite{oomu}など,授業をより効率的に進めることを目的としたシステムである.近年,Webを使用したWBT(Web-Based-Training)型\cite{wbt}やアプリケーション型\cite{takahasi}など,様々な種類の学習支援システムが開発・導入されている.これらのシステムには,授業内で使用する資料や課題を配布し回収する機能や出席確認を行う機能などが多数実装されている.そして,場所・時間を選ぶことなく課題に取り組めることや履歴情報を容易に管理することができるなどの生徒側・教師側の双方にメリットがある.これらのシステムは,授業を効率化するためのシステムであり生徒と教師の負担が減るという要素を持っている.これに対し,本研究ではより直感的で深い,史実の詳細な特徴や背景の理解を目的とし,イメージでの記憶を補助するアニメーションを用いる.

次に中等教育までの学習支援にARを使用した事例についていくつか述べる.小松ら\cite{komatsu}は,ARを用いた教材で天文分野における月の満ち欠けの学習支援システムを提案した.また,瀬戸崎ら\cite{setozaki}は,タブレット端末を利用した天体学習支援システムを開発した.これらの研究は,本研究同様に中等教育の学習支援を目的とし,ARや3DCGを用いて学習コンテンツを制作している.両者とも,紙テキストに記載されているARマーカを認識することで太陽・月・地球の3DCGをマーカ上に映し出し,視点移動などを行うことで学習内容が定着するかを確認するテストを行った.結果として,提案システムを用いた学習は両者とも平面図のみの学習よりも知識の定着に有効だったと示されている.しかし,両者の研究とも提案システムをアプリケーション形式で開発しているため,特定の学生しか利用できずユーザが限定されてしまう可能性が高い.このような事態に対処するため,本研究では様々な学生たちが各々の端末で利用できるようにWebアプリケーション形式を採用し,3次元CGを用いた歴史時系列アニメーションによる学習支援を提案した.


\section{提案システム}
\subsection{システム概要}

本システムは,中等教育の学習において歴史の学習を支援することを目的として,史実を把握し記憶させるために3次元CGを用いた時系列アニメーションをARで映し出す.そして,「場所法」と「ストーリーテリング法」という2種類の異なる記憶術を用いて,知識の定着を狙う.ユーザ自身のタブレット端末でARマーカを読み取り,ARマーカを基に3次元CGによるアニメーションを端末に表示させてイメージをとらえやすくすると同時に,史実の時系列をスライダーバーを用いてブラウジング可能にすることで,より史実を定着させやすくし学習支援を図る.本システムのイメージ図を,図\ref{fig:image3}に示す.また,本システムのフロー図を,図\ref{fig:flow2}に示す.


\subsection{システム構成}

ハードウェアの構成は,パソコン,ARマーカを付与した教材,タブレット端末からなる.パソコンは,CPUがintelCorei7であり,OSはWindows10がインストールされているモノを使用した.また,教材として,ARマーカである日本地図が表現されている紙のプリントを使用した.また,タブレット端末のOSはAndroidであり,インターネットブラウザとしてGoogleChromeがインストールされているモノを使用した.

ソフトウェアは,Unity,Blender,HTML5,Vuforiaを使用し作成した.その構成を図\ref{fig:software}に示す.

アニメーションを作成するソフトウェアはいくつか存在するが,本研究ではアニメーション内にAR出力やスライドバーなど複数の要素を組み込む必要があったため,Unityを使用する.プログラミング言語の一つであるC\#を用いて日本地図の色の変化や3次元CGの動作,史実の説明テキストの切り替え,AR出力などを制御した.今回使用した日本地図を図\ref{fig:chizu}に示す.本稿で用いた日本地図は旧国で分断し,それぞれの国をつなぎ合わせて作成した.

また,アニメーションによって勢力ごとに色が分けて表示されように実装した.説明テキストは,複数の史実のフォーマットを記述できるように,CSV(Comma Separated Value)ファイルの内容を出力した.直感的にアニメーションをブラウジング操作できるように2種類のスライドバーを用いて実装した.まず一つ目に,本研究での制作コンテンツである江戸時代前後の年代を指定するためのスライドバー(以下:年代バー)を配置した.年代バーを動かすことで,年代バーが示す時代の中ならどの年代のアニメーションでも選択できるようにした.二つ目に,アニメーションを制御するためのスライドバー(以下:アニメーションバー)を配置した.アニメーションバーを操作することで,ユーザが直感的に一時停止や巻き戻しができるように実装した.アニメーションの画面を図\ref{fig:pose_movie}に示す.


Blenderは,アニメーションの中に登場する3次元CGコンテンツを制作する際に使用した.本システムでは,史実に基づいた関連のある人物や建築物をアニメーションに登場させる.人物の3次元CGは,複数の異なる史実のアニメーションで使用できるように,色の変更を可能にすることで異なる登場人物を示せるようにした.また,様々な場所から遠征する様子を示すため,アニメーションの中で手足を動かして移動するように作成した.人物の3次元CGの例を図\ref{fig:busho}に示す.

建築物の3次元CGは,アニメーションの進行に応じて建設・倒壊時に演出を設けた.その3次元CGの例を図{fig:Castle}に示す.

これらの要素を加えることで,ユーザは史実に関連のある人物や建築物にさらなる親近感や躍動感などの印象を持つことができると考える.

学習用アニメーションコンテンツをブラウザ上に公開する際にHTML5を使用する.HTML5は,Webページの制作などに用いられる\cite{html}.初期版(HTML)より音声・動画・アニメーションなどのマルチメディアコンテンツ向けの機能が実装されているため,本研究ではこの言語を選択した.


%%Vuforiaは,Qualcomm社が提供するマルチプラットフォーム対応のAR制作用ライブラリであり,Unity上で動作する\cite{vuforia}.AR制作用ライブラリは,Vuforia以外に,ARToolkitなどが存在する.このライブラリは,加藤博一教授によって開発されたオープンソース型のAR構築ツールであり\cite{toolkit},単眼カメラと平面マーカを用いる.平面マーカとは,内部にカメラ識別用のパターンが描かれた正方形のマーカを表す.このマーカの位置姿勢などの情報を単眼カメラの映像からリアルタイムに計測することで,容易に3DCGをARとして出力することができる\cite{satou}.しかし,マーカの向きや一部隠蔽によって認識精度が著しく落ちてしまうという短所があるため,本研究ではARToolKitよりも認識精度が高く,様々な認識方式をもつVuforiaを選択した\cite{vuforia}.また,VuforiaはUnityと互換性が高くサンプルが多彩で使いやすいという点も選択の要因とした.本研究では日本地図をマーカとし,認識することで史実のアニメーションを端末の画面に映し出す.

ARを実装する際に,ARToolkit \cite{toolkit,satou}やVuforia \cite{vuforia} \footnote{Qualcomm社が提供するマルチプラットフォーム対応のAR制作用ライブラリであり,Unity上で動作する.}が考えられる.ARToolkitにおける平面マーカの利用も考えられるが,より頑健で認識方式が多様なVuforiaを利用することとした.本システムでは日本地図をVuforiaでのARマーカとして用い,地図上に戦国時代の史実が繰り広げられるようにした.日本地図のすべてに反応するわけではないが,汎用性のあるマーカの一つとしても地図は有効だと考えられる.


\section{実験}
\subsection{実験概要}
本研究では,時系列ブラウジング形式のアニメーションを視聴することによって,史実の流れや特徴,さらには他の史実との関係性などを分かりやすくかつ詳細に記憶定着させることを仮説とする.被験者は,19〜23歳の男女24名とする.本実験の条件として,2要因である.「3次元CG」を要因A(条件:4水準)とし,「時系列ブラウジング」(条件:3水準)を要因Bとし,2要因12条件を設定した.要因Aでは,3次元CGである戦国武将の有無,3次元CGと日本の旧国を繋ぐ線と3次元CGの移動の有無,の4水準を比較する.戦国武将を無しとする場合、球体の3次元CGを代用する.また要因Bでは,ビデオ形式,スライド形式,時系列ブラウジング形式,の3水準を比較する.2要因12条件はランダムに用意した.

\subsection{実験手順}
被験者に本研究で学習予定の江戸時代の戦乱・勢力図についての問題を4択形式で数問解答してもらう.ビデオ形式のアニメーションを5分ほど視聴する.視聴後に,問題番号や選択肢をランダムに変更した状態の4択形式の問題を解答する.次に,アニメーション内にある3次元CGを変更する.また,線や移動をなくす処理もほどこす.これを,要因Aと要因Bの条件を全て体験するまで続ける.最後に,システムの評価や歴史への興味をアンケートで解答する.


\subsection{実験}
実験参加者にはラテン方格法\cite{laten}を用いて,実験の評価項目を参考に作成した.



%% 考察は,制作したシステムの良点・悪点を取り上げながら改善点を述べる
\section{考察}

本研究では,中等教育で学習する歴史の史実を記憶定着させることを狙いとして,ARによる3次元CGアニメーションを用いた時系列ブラウジング形式の歴史学習支援システムを提案した.インタフェースとしてWebアプリケーションを適用したため,特殊なアプリケーションのインストールがなく誰でも利用可能なシステムであり,中高生の学習における障壁を下げることができたと考えられる.また,時系列ブラウジング形式のアニメーション表示を用いているため,ユーザが確認したい年代の史実を再生しながらつながりをもって学習できるため,複数の史実の関係を理解して記憶するうえで有用であると考えられる.

また,日本地図をARマーカとしているため,歴史の参考書・教科書など様々なコンテンツに導入できる.そのため,教科書などとの連携システムとして発展させることが期待される.教科書内の時代コンテンツに合わせた,地理的情報を含む史実のアニメーション化は,具体的にエピソードで記憶しやすくなることが期待できる.

現段階だと,本システムでは戦国時代の史実に関するアニメーションのみを扱ったため,各時代全ての勢力図を確認できるわけではない.また,戦乱に関する史実以外の文化的歴史に関しても取り扱っていない.しかし,戦乱だけではなく他の史実との関係を明確化することを目指し,今後は,戦乱以外のアニメーションを取り入れながら物語のようにブラウジングできるシステムを展開していくことを検討したい.例えば,地理的知識も史実と並行して学習できるような歴史時系列ブラウジングシステムを提案する.
また,アニメーションの内容を詳細に解説していないため,本システム単独での学習効果は低いと考えられ,学習効果を上げるためには,事前に参考書などで学習してから本システムを復習に用いてイメージ定着させることが望まれる.


\section{おわりに}

本研究では,ARアニメーションによる時系列ブラウジングを用いた歴史学習支援システムを提案した.生徒たちが自身の端末のカメラで日本地図をARマーカとして読み取ることで,アニメーションを史実の地理に基づいた地図上の位置に表示させ,わかりやすくイメージ化できることにより,歴史の学習意欲を上昇させ,さらなる史実の内容を能動的に学習することが期待される.

今回は,歴史の中でも江戸時代前後の戦乱に限定したが,今後は戦国の歴史に限らず,そのほかの時代の歴史や地理的情報の関係する学習コンテンツなど,様々な教科に発展させていく.例えば,過去に日本で発生した大地震の歴史を学習するコンテンツも検討中である.


%\ack%謝辞
\section*{謝辞}
本研究にご助力いただいた米澤ゼミの皆様に感謝いたします.また,実験に協力していただいた皆様に感謝いたします.


%参考文献

\bibliographystyle{junsrt}
\bibliography{sotsuron}
\end{document}

