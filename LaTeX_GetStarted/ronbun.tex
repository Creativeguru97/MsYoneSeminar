%
% このファイルはヒューマンインタフェース研究報告集用クラスファイル
% hisken.cls(ver.1.0) を利用した「原稿執筆の手引き」です.
% Revised in Jun. 7, 2007 by Yutaka Ishii
%

\documentclass[a4paper,dvipdfmx]{hisken}
%\usepackage[dviout]{graphicx}
%\usepackage[dvipdfm]{graphicx}
\usepackage{latexsym}
\usepackage{graphics}
\usepackage{epsfig} \let\epsfile=\epsfig
\usepackage{url}

%和文タイトル
\jtitle{環境音を利用した, 孤独感を軽減するエージェントの研究}

%著者日本名
\jauthor{
梅田 和希\thanks{関西大学 総合情報学部}   
}
%英文タイトル
\etitle{Research of virtual sound agent reduces feeling of loneliness}

%著者英文名
\eauthor{
Kazuki Umeda\thanks{Kansai University, Faculty of Informatics},
}
% 名前は,はじめの一文字だけ大文字にしてください.

\begin{document}

%maketitle は abstract と keyword の後に入れてください.

\begin{abstract}
本研究では, 孤独感を現代の先進国にある社会問題の一つとして取り上げ, それを軽減する手法として音で人間の存在感を模倣した仮想エージェントを提案する. そしてその存在に対して人間が感情移入をして, エージェントが人間の存在を代替し得るのか否かを検討する. その際, 孤独という概念は非常に主観的なものであり, また孤立などの密接に関係する近い概念と混同しやすい. よってまず初めに, 心理学の分野でそれらの定義や区別を試みた研究や調査を取り上げ, 本研究で取り扱う範囲を明確にする. 次に孤独が人間に及ぼす影響, そして主要な先進国に行われた孤独に関係する調査を取り上げる. そして最後に工学分野での関連研究や製品と本研究でのエージェントについて述べる.
\end{abstract}

\begin{keyword}	
%%History, augmented reality, 3DCG, animation, learning support
孤独, 孤立, loneliness, 音, アンビエント, 存在感, キャラクター, 共感, 共感的行動, 協働, RNN, LSTM
\end{keyword}

\maketitle
	
\section{孤独の定義と扱う範囲}
孤独という概念は定義が曖昧なまま、漠然としたイメージで日常で述べられることが多い. しかしその実態は単一の心理状態ではなく, 孤立などの一見混同されやすい概念と複雑に重なりあっている. 解決する課題を明確にするため, まずここに心理学の分野で孤独とその周辺の概念がどのように定義されているのかを取り上げる。
\subsection{孤立と孤独}

CornwellとWaiteによる2009年の調査では, 客観的に社会からどの程度孤立しているかを示すSocial disconnectedness(社会的孤立)と, 主観的に交友, 交際関係や支援される対象に不足を感じているかを示すPerceived isolation(認知された孤立)にわけた{}\cite{tex1}. そしてそのそれぞれが精神的, 身体的にどの程度影響を与えるのかを調べた.(孤独については心理学的アプローチと, 社会学的アプローチがこれまでになされている. 前者は被験者の主観的な孤立の感覚に注目し, 後者が客観的に数量で表される指標で測ろうとするが, CornwellとWaitは, これまでその2つの軸を同時に扱って比較されていなかったことを課題とした.)
2014年にJohn T. CacioppoとStephanie Cacioppoによって行われた調査においても、社会的孤立と健康状態の関係を調べる上で、社会的孤立はObjective isolation(客観的孤立: 家族,や友人の数など, 数量化できる要素が要因)とSubjective isolation(主観的孤立)の2つの側面を持つとした{}\cite{tex2}。さらにSubjective isolationを前述のPerceived isolationという言葉で言い換え、それを孤独と同義として扱った。

\subsection{2種類の孤独}

また2018年にアメリカ人の18~70歳の成人1800人を対象に行われた調査では, 孤独はSocial loneliness(社会的孤独)とEmotional loneliness(感情的孤独)の二つの軸に分けて考えられた{}\cite{tex3}。
Social loneliness(社会的孤独)とは, 自分の周囲にいる人の数に理想とのギャップを感じている状態である. 自分には親, 友達, また困った時に助けてくれる間柄の人がほとんどいないなどと感じていることが考えられる. それに対しEmotional loneliness(感情的孤独)とは, 関係を持つ人の数でではなくその質に満足していない状態を表す. 家庭を持っていたり, 多数の社会的繋がりを持っていても心理的な距離を感じたり, その繋がりが本人を消耗させる原因になっている場合も考えられる. 
この2つの軸から, 被験者の孤独感は以下の4つのサブクラスに分けられた. 
\begin{itemize}
\item Low loneliness: どちらの孤独感も主観的度合いが低い
\item Social loneliness: 社会的繋がりの量(数)に不満足である
\item Emotional loneliness:  社会的繋がりの質に不満足である
\item Social and emotional loneliness: 社会的繋がりの量(数), 質共に不満足である
\end{itemize}
このSocial loneliness(社会的孤独)は, CornwellとWaite(2009)やJohn T. CacioppoとStephanie Cacioppo(2014)らのいうPerceived isolation(本人が認知している孤立)と同様の概念を指すと思われる。

他にも多数の孤独や孤立を定義して調査した研究が存在するが, それらを集約すると以下の3つに分類されることが多いと考えた.
\begin{itemize}
\item Social Disconnectedness / Objective isolation (以下, 社会的孤立状態とする.)
\item loneliness (以下, 孤独感とする. 2と同義とされることもある.)
\begin{enumerate}
\item Social loneliness(以下, 社会的孤独とする. Perceived isolationおよびSubjective isolationと同義と考えられる.)
\item emotional loneliness(以下, 感情的孤独とする.)
\end{enumerate}
\end{itemize}

\section{}

\section{先進諸国と日本の現状}
アメリカの, 国際的に医療と健康のサービスを展開する企業であるCignaが2018年の5月に発表した, 20,096人のアメリカ在住の18歳以上の人を対象に行った大規模な調査の結果(*5)がある, その一部に次の様なものがある. 
\begin{itemize}
\item 約半数の人が時々, または慢性的に孤独感を感じている(46%), また周囲から取り残されていると感じている(47%)
\item 約27%の人が自分のことを理解している人が全く, または殆どいないと感じている. 
\item 約43%の人が, 既にある人間関係に意義を感じていない, または周囲から孤立していると感じている. 
\item 他の人と共同生活をしている人(UCLA孤独スコア: 43.5) は, 一人で生活している人(UCLA孤独スコア: 46.4)に比べて孤独感が比較的軽い. しかし一人でなくても, 本人が片親や一人で身元引き受け人になっている場合, 子供と一緒に暮らしているにも関わらずより大きな孤独感(UCLA孤独スコア: 48.2)を感じている. 
\end{itemize}

この結果から, アメリカにおいて孤独や孤立の感覚は大多数の割合の国民に及ぶ社会問題と考えるのが妥当であると考える.

また, 同年10月にイギリスのBBCの主要なラジオ放送局の一つであるBBC Radio 4が行った. オンライン上での孤独に関する調査結果(*6, *7)がある(国内のマンチェスター大学、ブルネル大学、エクセター大学の協力の元に行われた). 同調査では約55,000人の一般人からの回答が集まり, 孤独に関するこの類の調査としては世界最大規模となった. 
その結果, 16~24歳の年齢層が,他の年齢層に比べて最も孤独感を感じている人の割合が多いことが分かった(孤独感を時々/頻繁に感じると回答した割合が40%と, 最も多かった. 65~74歳は29%, 75歳以上は27%であった. ). 

日本では近年において, 国内で一般人を対象とした,孤独に関する信頼できる調査が少ないが, ”引きこもり”についての調査が存在する. 
内閣府は2009年と2015年に若年層(15~39歳)の, また2015年に中高年層(40~64歳)の引きこもりに該当する国内の人数を推定した調査(*9, *10)を行っている. その結果, 若年層で54.1万人, また中高年層で61.3万人と推定された.
このことから, 我が国でもSocial isolation(社会的孤立)に相当する状態にある人が多数いると思われる.


\section{本研究で取り扱う範囲}
emotional lonelinessを抱くシチュエーションは様々であり, それぞれで解決可能性や解決方法が異なる. しかしその多くは次の3つのタイプに分類できると考えた.

\begin{itemize}
\item 不協和: これはすでに持っている人間関係が, 本人の理想とする状態から離れている時に起こりやすい. 例えば, 学校で仲間集団からいじめを受けている状況や, 恋人や夫婦間などの親密な相手との関係が原因で精神的に疲弊している場合などである. この時, 周囲に助けを求めたり, 相談できる他者がいない場合にemotional lonelinessを抱くと考えられる.
\item 欠乏: これは自分が持っていない他者との関係を欲している状況で起こりやすい. 異性との親密な関係を欲していたり, 仕事の悩みを相談できる相手がいないといった状況などが考えられる.
\item 喪失: これはすでに持っている他者との関係が何らかの要因で失われた, または相対的にこれまでより極めて薄くなった時に起こりやすい. 長年連れ添ったパートナーとの死別や, 留学したことにより家族と長期間離れているような状況が要因として考えられる.
\end{itemize}

上の3つのシチュエーションのうち, 不協和に関しては既存の特定の個人, または集団間との人間関係の質の問題が要因となっており, そのエージェントによる解決策には, 人間関係を仲介する役割を担うものが考えられる. また喪失に関しては, エージェントによって既存の個人の存在の代替を担うことは現時点では考えにくいが, 特定の個人についてのポジティブな感情を思い起こさせる役割を持つエージェントなどが考えられる. しかし欠乏というシチュエーションの場合, 関係を欲する対象はしばしば不特定である場合がある. 物心がついた時から肉親がいないため, 授業参観日にクラスメイトが羨ましくなるといったものや, 親友の異性関係を見て, 同様の存在を欲するといったケースがそうである. これらのケースでは, 欲する人間関係の携帯のモデルは存在するが具体的な個人または集団は存在しない. よってここに擬人化された仮想エージェントが人間の存在を代替する可能性があると考えた. よって本研究では欠乏に焦点を絞って擬人化エージェントの属性や人格的特徴を作っていきたい.


\section{提案システム}
\subsection{システム概要}
欠乏し得る人間関係は個人によって様々であり, それぞれについてエージェントの性格的属性や内部の心理状態, 身体的特徴などを変える必要がある. また, 欲する関係の親密度によって, 要求されるそれらの作りこみの度合いも変化する. 

本研究では浅い関係である

\subsection{システム構成}


\section{実験}
\subsection{実験概要}

\subsection{実験手順}




%% 考察は,制作したシステムの良点・悪点を取り上げながら改善点を述べる
\section{考察}


\section{おわりに}

%\ack%謝辞
\section*{謝辞}
本研究にご助力いただいた米澤ゼミの皆様に感謝いたします.また,実験に協力していただいた皆様に感謝いたします.

\begin{thebibliography}{99}
\bibitem{tex1}
	Knuth,D.E.:
	\TeX{}ブック,
	アスキー出版局,(1989).
\bibitem{tex2}
	奥村:
	\LaTeX{}美文書作成入門,
	技術評論社,(1991).
\bibitem{tex3}
    奥村:
    \LaTeX2$\varepsilon$美文書作成入門(改訂第4版),
    技術評論社,(2006).
\end{thebibliography}


%参考文献

\bibliographystyle{junsrt}
\bibliography{sotsuron}
\end{document}

