%
% このファイルはヒューマンインタフェース研究報告集用クラスファイル
% hisken.cls(ver.1.0) を利用した「原稿執筆の手引き」です.
% Revised in Jun. 7, 2007 by Yutaka Ishii
%

\documentclass[a4paper,dvipdfmx]{hisken}
%\usepackage[dviout]{graphicx}
%\usepackage[dvipdfm]{graphicx}
\usepackage{latexsym}
\usepackage{graphics}
\usepackage{epsfig} \let\epsfile=\epsfig
\usepackage{url}

%和文タイトル
\jtitle{ARアニメーションを用いた戦国歴史学習支援システムの開発}

%著者日本名
\jauthor{
田端 剛\thanks{関西大学 総合情報学部}   
}
%英文タイトル
\etitle{History Learning Support System on the Age of Provincial Wars with AR Animation on a Map}

%著者英文名
\eauthor{
Tsuyoshi Tabata\thanks{Kansai University, Faculty of Informatics},
}
% 名前は,はじめの一文字だけ大文字にしてください.

\begin{document}

%maketitle は abstract と keyword の後に入れてください.

\begin{abstract}
本研究では,ARを用いたアニメーションを開発し,史実の流れや関係性を視覚的に提示して中等教育の歴史の理解を助ける学習支援を行う.本システムでは,歴史教育の中でも特に江戸時代前後に発生した戦いに関する史実に重点を置き,AR,3次元CGによる地図上の史実の視覚化と,スライドバーによる時系列ブラウジングシステムを提案する.これにより,あらゆる歴史教育用コンテンツをブラウジングできるように記憶定着することを狙う.CGは,各史実に関係性のある人物や建築物を示す際に用いる.また,スライドバーは,史実の年代ごとのアニメーションを時間ブラウジング操作できるように実装した.中等教育で学習する史実の知識を定着させる効果を予想し被験者実験を行ったところ,○○という結果が示された.
\end{abstract}

\begin{keyword}	
%%History, augmented reality, 3DCG, animation, learning support
歴史, AR, 3次元CG, アニメーション, 学習支援
\end{keyword}

\maketitle
	
\section{はじめに}
\subsection{ARの現状}
本研究では,ARを用いたアニメーションを開発し,史実の流れや関係性を視覚的に提示して中等教育の歴史の理解を助ける学習支援を行う.本システムでは,歴史教育の中でも特に江戸時代前後に発生した戦いに関する史実に重点を置き,AR,3次元CGによる地図上の史実の視覚化と,スライドバーによる時系列ブラウジングシステムを提案する.これにより,あらゆる歴史教育用コンテンツをブラウジングできるように記憶定着することを狙う.CGは,各史実に関係性のある人物や建築物を示す際に用いる.また,スライドバーは,史実の年代ごとのアニメーションを時間ブラウジング操作できるように実装した.中等教育で学習する史実の知識を定着させる効果を予想し被験者実験を行ったところ,○○という結果が示された.

\section{実験}
\subsection{実験概要}
本研究では,時系列ブラウジング形式のアニメーションを視聴することによって,史実の流れや特徴,さらには他の史実との関係性などを分かりやすくかつ詳細に記憶定着させることを仮説とする.被験者は,19〜23歳の男女24名とする.本実験の条件として,2要因である.「3次元CG」を要因A(条件:4水準)とし,「時系列ブラウジング」(条件:3水準)を要因Bとし,2要因12条件を設定した.要因Aでは,3次元CGである戦国武将の有無,3次元CGと日本の旧国を繋ぐ線と3次元CGの移動の有無,の4水準を比較する.戦国武将を無しとする場合、球体の3次元CGを代用する.また要因Bでは,ビデオ形式,スライド形式,時系列ブラウジング形式,の3水準を比較する.2要因12条件はランダムに用意した.

\subsection{実験手順}
被験者に本研究で学習予定の江戸時代の戦乱・勢力図についての問題を4択形式で数問解答してもらう.ビデオ形式のアニメーションを5分ほど視聴する.視聴後に,問題番号や選択肢をランダムに変更した状態の4択形式の問題を解答する.次に,アニメーション内にある3次元CGを変更する.また,線や移動をなくす処理もほどこす.これを,要因Aと要因Bの条件を全て体験するまで続ける.最後に,システムの評価や歴史への興味をアンケートで解答する.


\subsection{実験}
実験参加者にはラテン方格法を用いて,実験の評価項目を参考に作成した.



%% 考察は,制作したシステムの良点・悪点を取り上げながら改善点を述べる
\section{考察}

本研究では,中等教育で学習する歴史の史実を記憶定着させることを狙いとして,ARによる3次元CGアニメーションを用いた時系列ブラウジング形式の歴史学習支援システムを提案した.インタフェースとしてWebアプリケーションを適用したため,特殊なアプリケーションのインストールがなく誰でも利用可能なシステムであり,中高生の学習における障壁を下げることができたと考えられる.また,時系列ブラウジング形式のアニメーション表示を用いているため,ユーザが確認したい年代の史実を再生しながらつながりをもって学習できるため,複数の史実の関係を理解して記憶するうえで有用であると考えられる.

また,日本地図をARマーカとしているため,歴史の参考書・教科書など様々なコンテンツに導入できる.そのため,教科書などとの連携システムとして発展させることが期待される.教科書内の時代コンテンツに合わせた,地理的情報を含む史実のアニメーション化は,具体的にエピソードで記憶しやすくなることが期待できる.

現段階だと,本システムでは戦国時代の史実に関するアニメーションのみを扱ったため,各時代全ての勢力図を確認できるわけではない.また,戦乱に関する史実以外の文化的歴史に関しても取り扱っていない.しかし,戦乱だけではなく他の史実との関係を明確化することを目指し,今後は,戦乱以外のアニメーションを取り入れながら物語のようにブラウジングできるシステムを展開していくことを検討したい.例えば,地理的知識も史実と並行して学習できるような歴史時系列ブラウジングシステムを提案する.
また,アニメーションの内容を詳細に解説していないため,本システム単独での学習効果は低いと考えられ,学習効果を上げるためには,事前に参考書などで学習してから本システムを復習に用いてイメージ定着させることが望まれる.


\section{おわりに}

本研究では,ARアニメーションによる時系列ブラウジングを用いた歴史学習支援システムを提案した.生徒たちが自身の端末のカメラで日本地図をARマーカとして読み取ることで,アニメーションを史実の地理に基づいた地図上の位置に表示させ,わかりやすくイメージ化できることにより,歴史の学習意欲を上昇させ,さらなる史実の内容を能動的に学習することが期待される.

今回は,歴史の中でも江戸時代前後の戦乱に限定したが,今後は戦国の歴史に限らず,そのほかの時代の歴史や地理的情報の関係する学習コンテンツなど,様々な教科に発展させていく.例えば,過去に日本で発生した大地震の歴史を学習するコンテンツも検討中である.


%\ack%謝辞
\section*{謝辞}
本研究にご助力いただいた米澤ゼミの皆様に感謝いたします.また,実験に協力していただいた皆様に感謝いたします.


%参考文献

\bibliographystyle{junsrt}
\bibliography{sotsuron}
\end{document}

