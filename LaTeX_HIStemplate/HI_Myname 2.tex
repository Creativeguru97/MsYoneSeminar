%
% このファイルはヒューマンインタフェース研究報告集用クラスファイル
% hisken.cls(ver.1.0) を利用した「原稿執筆の手引き」です.
% Revised in Jun. 7, 2007 by Yutaka Ishii
%

\documentclass[a4paper,dvipdfmx]{hisken}
%\usepackage[dviout]{graphicx}
%\usepackage[dvipdfm]{graphicx}
\usepackage{latexsym}
\usepackage{graphics}
\usepackage{epsfig} \let\epsfile=\epsfig
\usepackage{url}

%和文タイトル
\jtitle{}

%著者日本名
\jauthor{
 \thanks{関西大学 総合情報学部}   
}
%英文タイトル
\etitle{}

%著者英文名
\eauthor{
 \thanks{Kansai University, Faculty of Informatics},
}
% 名前は,はじめの一文字だけ大文字にしてください.

\begin{document}
\[%maketitle は abstract と keyword の後に入れてください.

\begin{abstract}

\end{abstract}

\begin{keyword}	
%%5つまで(日本語)
・・, ・・, ・・, ・・, ・・
\end{keyword}

\maketitle
	
\section{はじめに}


\section{関連研究}


\section{提案システム}


\section{実験}


\section{考察}


\section{おわりに}


%\ack%謝辞
\section*{謝辞}
本研究にご助力いただいた米澤ゼミの皆様に感謝いたします.また,実験に協力していただいた皆様に感謝いたします.


%参考文献

\bibliographystyle{junsrt}
\bibliography{sotsuron}
\]
\end{document}